% The following configuration file is greatly basen on a configuration file that was (as far as I know) developed by Waqas Abbas for his Master Thesis at TU-Darmstadt in 2017.

\usepackage{hyperref}
\usepackage{emptypage}
\usepackage[toc,page]{appendix}
% Handling chapters & section title alignment

% no indentation for new paragraphs
%\setlength{\parindent}{0pt}

% input encoding
\usepackage[utf8]{inputenc}

% font encoding
\usepackage[T1]{fontenc}

% german or english document
% \usepackage[ngerman]{babel}
\usepackage[english]{babel}

% set line spacing to 1.5
\usepackage{setspace}
\onehalfspacing

\usepackage{listings} %code highlighter
\lstset{
	literate={ö}{{\"o}}1
	{ä}{{\"a}}1
	{ü}{{\"u}}1
}
\usepackage{color} %use color
\definecolor{mygreen}{rgb}{0,0.6,0}
\definecolor{mygray}{rgb}{0.5,0.5,0.5}
\definecolor{mymauve}{rgb}{0.58,0,0.82}

\usepackage[T1]{fontenc}
\usepackage{selinput}
\SelectInputMappings{%
	adieresis={ä},
	eacute={é},
	Lcaron={Ľ},
}

%Customize the look
\lstset{ %
	backgroundcolor=\color{white}, % choose the background color; you must add \usepackage{color} or \usepackage{xcolor}
	basicstyle=\footnotesize, % the size of the fonts that are used for the code
	breakatwhitespace=false, % sets if automatic breaks should only happen at whitespace
	breaklines=true, % sets automatic line breaking
	captionpos=b, % sets the caption-position to bottom
	commentstyle=\color{mygreen}, % comment style
	deletekeywords={...}, % if you want to delete keywords from the given language
	escapeinside={\%*}{*)}, % if you want to add LaTeX within your code
	extendedchars=true, % lets you use non-ASCII characters; for 8-bits encodings only, does not work with UTF-8
	frame=single, % adds a frame around the code
	keepspaces=true, % keeps spaces in text, useful for keeping indentation of code (possibly needs columns=flexible)
	keywordstyle=\color{blue}, % keyword style
	% language=Octave, % the language of the code
	morekeywords={*,...}, % if you want to add more keywords to the set
	numbers=left, % where to put the line-numbers; possible values are (none, left, right)
	%numbersep=5pt, % how far the line-numbers are from the code
	numberstyle=\tiny\color{mygray}, % the style that is used for the line-numbers
	xleftmargin=2.5em,
	xrightmargin=1em,
	framexleftmargin=2em,
	rulecolor=\color{black}, % if not set, the frame-color may be changed on line-breaks within not-black text (e.g. comments (green here))
	showspaces=false, % show spaces everywhere adding particular underscores; it overrides 'showstringspaces'
	showstringspaces=false, % underline spaces within strings only
	showtabs=false, % show tabs within strings adding particular underscores
	stepnumber=1, % the step between two line-numbers. If it's 1, each line will be numbered
	stringstyle=\color{mymauve}, % string literal style
	tabsize=2, % sets default tabsize to 2 spaces
	title=\lstname % show the filename of files included with \lstinputlisting; also try caption instead of title
}

%javalanguage
\definecolor{darkgray}{rgb}{.4,.4,.4}
\definecolor{purple}{rgb}{0.65, 0.12, 0.82}

%define Javascript language
\lstdefinelanguage{JavaScript}{
	keywords={typeof, new, true, false, catch, function, return, null, catch, switch, var, if, in, while, do, else, case, break},
	keywordstyle=\color{blue}\bfseries,
	ndkeywords={class, export, boolean, throw, implements, import, this},
	ndkeywordstyle=\color{darkgray}\bfseries,
	identifierstyle=\color{black},
	sensitive=false,
	comment=[l]{//},
	morecomment=[s]{/*}{*/},
	commentstyle=\color{purple}\ttfamily,
	stringstyle=\color{red}\ttfamily,
	morestring=[b]',
	morestring=[b]"
}

\lstset{
	language=JavaScript,
	extendedchars=true,
	basicstyle=\footnotesize\ttfamily,
	showstringspaces=false,
	showspaces=false,
	numbers=left,
	numberstyle=\footnotesize,
	numbersep=9pt,
	tabsize=2,
	breaklines=true,
	showtabs=false,
	captionpos=b
}


% multiple rows inside a table
\usepackage{multirow}


% consistent text for graphics
\usepackage{psfrag}

% bibliography with biblatex and biber backend
\usepackage[style=alphabetic,backend=biber]{biblatex}

% same style for URLs
\urlstyle{same}

% title not italic
\DeclareFieldFormat{title}{#1\isdot}

% colon between author and title
\renewcommand{\labelnamepunct}{\addcolon\space}

% use semicolon when multiple authors
\renewcommand{\multinamedelim}{\addsemicolon\space}

% seperate last author with semicolon
\renewcommand{\finalnamedelim}{\addsemicolon\space}

% format name
%\DeclareNameFormat{default}{\usebibmacro{name:last-first}{#1}{#4}{#5}{#7}\usebibmacro{name:andothers}}

% text for url
\DefineBibliographyStrings{ngerman}{urlseen={abgerufen am}}

% biblatex

\addbibresource{bibliography.bib}

% properly underlined url in bibliography
\usepackage{url}

% adds align command
\usepackage{amsmath}

% use graphics
\usepackage{graphicx}
\graphicspath{{gfx/}}
%\setinstitutionlogo{Images/dsp_logo.jpg}

% display pics next to each other, don't use older subfigure package
\usepackage{subfig}

% use nicer tables with footnotes
\usepackage{ctable}
\setupctable{
captionskip=0pt, framerule=0pt, nostar,
center, framesep=0pt, pos=tbp,
continued=(continued), maxwidth=0pt, super,
doinside={}, mincapwidth=0pt, table,
framebg=1 1 1, nonotespar, botcap,
framefg=0 0 0, nosideways, width=0.9\textwidth
}

% force all floats to render with \FloatBarrier
\usepackage{placeins}

% acronyms and nomenclature
\usepackage[acronym, toc, nonumberlist]{glossaries}

% links within latex


% break long URLs in toc
\usepackage{breakurl}

\usepackage{array}
\newcolumntype{L}[1]{>{\raggedright\let\newline\\\arraybackslash\hspace{0pt}}m{#1}}
\newcolumntype{C}[1]{>{\centering\let\newline\\\arraybackslash\hspace{0pt}}m{#1}}
\newcolumntype{R}[1]{>{\raggedleft\let\newline\\\arraybackslash\hspace{0pt}}m{#1}}

\setcounter{secnumdepth}{3}
\setcounter{tocdepth}{2}

% set \emph to display text italic
\DeclareTextFontCommand{\emph}{\textit}

%checkmarks for table
%\usepackage{amssymb}% http://ctan.org/pkg/amssymb
\usepackage{pifont}% http://ctan.org/pkg/pifont
\newcommand{\myes}{\ding{51}}%
\newcommand{\mno}{\ding{55}}%
\renewcommand{\emph}{\textit}%
