Since Reactive Programming was introduced in 1997 it has gained in popularity and today there is an implementation for the paradigm in most popular programming languages. However, the development tool support, including Integrated Development Environments and debuggers, is still very limited. Traditional debuggers are designed to debug imperative and sequential code and, therefore, are not suitable to use with Reactive applications. Although over the years some tools for specific languages were developed, there are currently only two, that we are aware of, which target Reactive libraries and frameworks for JavaScript. Even though JavaScript is placed 8th this month in the TIOBE Index \cite{LanguageIndex}. The two existing tools are the Chrome Reactive Inspector and RxFiddle.
In this thesis we continue the work on the Chrome Reactive Debugger and advance it on the path to being usable in a production environment. The Chrome Reactive Inspector provides many features to help users understand how dependencies between entities in the Reactive application are formed at runtime and to find logical errors within its source code. In order to increase the usability of the Chrome Reactive Inspector, we implemented means to connect the abstract representation with corresponding source code and improved the User Interface. We also increased the tool's computational performance which helps cope with resource demanding applications.