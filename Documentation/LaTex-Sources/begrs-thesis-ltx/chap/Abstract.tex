Since its introduction in 1997, reactive programming has become increasingly popular. Today there is an implementation of the paradigm in most popular programming languages. However, the development tool support, including Integrated Development Environments and debuggers, is still very limited. Traditional debuggers are designed to debug imperative and sequential code and are therefore not suitable to use with reactive applications. Although some development tools have already been developed for specific languages over the years, there are currently only two that we are aware of which target reactive libraries and frameworks for JavaScript. This is a rather surprising fact considering that JavaScript is currently number 8th of the most used programming languages according to the TIOBE Programming Community Index \cite{LanguageIndex}. The two existing tools are the Chrome Reactive Inspector and RxFiddle.
In this thesis, we continue the work on the Chrome Reactive Inspector and advance it with regards to being usable in a production environment. The Chrome Reactive Inspector provides many features to help users understand how dependencies between entities in a reactive application are formed at runtime and find logical errors within this applications source code. In order to increase the usability of the Chrome Reactive Inspector, we implement means to connect the abstract representation with corresponding source code and we improve the User Interface. We also increase the tool's computational performance, which helps cope with resource demanding applications.