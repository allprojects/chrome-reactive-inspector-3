\chapter{Conclusion and Future}
\label{ch:Future}
% conclusion and future

% Conclusion:
% - CRI supports many additional features that help to cope with some of the issues encountered by reactive debuggers that are currently not present in RxFiddle.
% - Event though many improvements have been made to cope with large*(always use the same terminology) graphs/histories/applications like performance improvements, it does not cover everything.
% - The UI has been updated to direct the users focus and speed up or reduce cognitive load for some user tasks.
% - To strengthen links* between abstract representation and acutal code, source code previews* have been introduced.
% - CRI has a way to go before being able to properly use in a production environment - Typescript, ecma6 support see future.
% - 

% Future:

% ecma6 import support and bundled Rx/Bacon detection, + typescript
	% currently searches for <script Rx.js/Bacon.js>
	% if all used libraries are bundled and only own project files are not
	% will not work
	% all module/file loaders are not supported yet e.g. require.js since module loading prior to ecma6 is a diverse field
	% and most applications use one this is one of the main obstacles before the extension can be used by a wider audience (see temp_references)
	% ecma6 import references file not necessarily in html as <script>
	%TODO check if above is true
	%TODO check if typescript compilatino to un-bundled js is possible
	%- the breach of the isolated world of chrome context scripts enables the developer to have a total access to the pages javascript 	which can be used to solve the problem of "import" statements. The developer could overwrite the import/require/.. comands of each known framework and provide a custom function that does the loading in additional 
	% add nodejs support
	% - as long as all the files that are desired for debugging are present, it does not matter if they are loaded via a module loader, because the file can be instrumented seperately. The only problem is to find the file, a simple hook in a specially designed comment may be used <!-- cri: src="path to jsfile" --> that is ignored in a production environment. The problem is to access the shared Rx/Bacon object as a simple comment that specifies Rx or Bacon would not suffice.
	
			% instrumenting files via scanning the html for script tags
	% -> will not work for module loaders like require.js or ecma6
	% -> will not work with bundled JavaScript code - but since in development there should be a non bundled version available not 
	%TODO: verify this:
	% a big problem in JavaScript. But for TypeScript since the extension does not support it and when compiling there may already be bundling in place. (#add some text from future here#)

% Solution management:
% - enabled storage of project specific configuration. 
% - necessary for fine grained recording to be effective as described in [Increased recording complexity]

 	
