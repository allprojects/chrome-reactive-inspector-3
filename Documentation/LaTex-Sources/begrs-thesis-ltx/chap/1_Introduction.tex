\chapter{Introduction} \label{ch:Introduction}
The Chrome Reactive Inspector is an advanced debugging tool for Web applications using reactive programming with the libraries RxJS or BaconJS in JavaScript. The tool provides many features similar to those of traditional debuggers, such as breakpoints, tailored especially to the needs of users debugging reactive programming systems for these two JavaScript libraries mentioned above. This chapter gives a short introduction to reactive programming in general and the Chrome Reactive Inspector in particular.

\section{Background}
The reactive programming paradigm evolved from the Observer Design Pattern and therefore supports the publishing of changes to an object to a list of subscribers. While the Observer Design Pattern only supports returning one value at a time, in reactive programming it is possible for an object to return multiple values over time. Reactive programming provides means to specify dependencies between its entities in a declarative way. It is designed to handle event and data streams. This includes application User Interfaces and message passing in distributed system. Reactive programming is difficult to debug without an abstract representation of dependencies and streams.
While there are already many implementations of the reactive programming paradigm for JavaScript, there are currently only two development tools designed to handle debugging in applications that use reactive programming in JavaScript: RxFiddle and the Chrome Reactive Inspector. RxFiddle supports the reactive library RxJS and the Chrome Reactive Inspector supports the libraries RxJS and BaconJS. Both tools enables advantageous debugging for reactive JavaScript applications without the need to modify the source code. The Chrome Reactive Inspector is a Google Chrome extension based on the extension Reactive Inspector for the Scala IDE for Eclipse. Being a Google Chrome extension provides the tool with platform independence, presumed a version of Google Chrome that supports extensions runs on the platform.
The Chrome Reactive Inspector displays an abstract representation of the inspected application in form of a dependency graph. It also provides the user with the option to examine the creation of that graph in addition to tracking value updates traveling along the chain of dependencies at any time during the application's execution. In contrast,  RxFiddle uses a marble diagram as the abstract representation.

\section{Motivation}
The Chrome Reactive Inspector is one of only two tools that support developers with debugging reactive JavaScript applications. Advancing it serves the purpose of improving debugging experience for developers and reducing the need to modify the source code to enable a less sophisticated approach. Our goal is to increase usability of the Chrome Reactive Inspector and to advance its features so that it will eventually be usable in a production environment without limitation. The first step is to merge the previous work on the Chrome Reactive Inspector by Waqas Abbas and Pradeep Baradur. We then target the need for more detailed information about individual nodes of the dependency graph. We further aim at improving the performance and User Interface limitations of the Chrome Reactive Inspector when dealing with specific types of Web applications in order to cover a wider range of supported applications. In addition we examine the extension's User Interface in general with regards to a sound design, the required learning time and, eventually, their speed performing debugging tasks.

\section{Our Contribution}

In the last two sections, we presented a brief overview of reactive programming, introduced the Chrome Reactive Inspector, its origin and the scope of this thesis.
In summary, we make the following contributions:

\begin{itemize}
	\item Merging previous work on the Chrome Reactive Inspector
	\item Reworking the User Interface, including the reduction of cognitive load on the user, to better support them in performing debugging tasks
	\item Providing additional details to the dependency graph by connecting nodes with their corresponding lines of code
	\item Reducing computational resource consumption when working with applications that contain rapidly updated observables
	\item Providing a baseline for additional development targeting excessively created observables
\end{itemize}

\section{Outline}
In chapter \ref{ch:StateOfTheArt} we explain reactive programming in general and the two JavaScript libraries RxJS and BaconJS in detail. We also examine existing tools for debugging reactive systems and why traditional debuggers are not suitable for this task. In addition, we present the previous work on the Chrome Reactive Inspector as well as its main competitor RxFiddle.
Our contributions and the reasoning behind choosing specific approaches over other possible solutions is explained in chapter \ref{ch:Contribution}.
Chapter \ref{ch:Evaluation} contains the verification of our claims and implementations as well as the results of the evaluation of the most important features in the Chrome Reactive Inspector for a set of test applications.
In chapter \ref{ch:Future} we summarize the result of this thesis and also point out possible targets for further development to increase the applicability of the Chrome Reactive Inspector.
