\chapter{Introduction} \label{ch:Introduction}
This chapter gives a short introduction to Reactive Programming in general and the Chrome Reactive Inspector. The Chrome Reactive Inspector is an advanced debugging tool for systems using Reactive Programming in JavaScript with the libraries RxJS or BaconJS. The tool provides many features, similar to those of traditional debuggers like Breakpoints, tailored especially to the needs of users debugging Reactive Programming Systems for these two JavaScript libraries.

\section{Background}
The Reactive Programming paradigm is an evolution of the Observer design pattern. In both, changes to an object are published to a list of subscribers, but in Reactive Programming the object is able to return multiple values over time while the observer design pattern only supports one value at a time. Reactive Programming provides means to specify dependencies between its entities in a declarative way. Reactive Programming is designed to handle event and data streams. This includes for example application User Interfaces and message passing in distributed system. Due to its nature, Reactive Programming is hard to debug without an abstract representation of the dependencies or streams.
For JavaScript there are many implementations of the Reactive Programming paradigm. There are currently only two tools designed to handle Reactive Programming in JavaScript that we are aware of. RxFiddle, which only supports the Reactive library RxJS and the Chrome Reactive Inspector, which only supports the libraries RxJS and BaconJS. Both tools enables advantageous debugging for Reactive JavaScript applications without the need to modify the source code. The Chrome Reactive Inspector is a Google Chrome extension based on the Scala IDE for Eclipse extension Reactive Inspector. Being a Google Chrome extension provides the tool with platform independence, presumed a version of Google Chrome that supports extensions runs on the platform.
The Chrome Reactive Inspector displays an abstract representation in form of a Dependency Graph for the inspected application. It also provides the user with the possibility to examine the creation of that graph in addition to tracking value updates traveling along the chain of dependencies at any time during the application's execution.%TODO possibility?
In contrast RxFiddle uses a Marble Diagram as the abstract representation.

\section{Motivation}
The Chrome Reactive Inspector is one of only two tools that support developers in debugging Reactive applications. Advancing it serves the purpose of improving debugging experience for developers and to reduce the need of modifying the source code in order of enabling a less sophisticated approach. Our goal is to increase usability of the Chrome Reactive Inspector and to advance its features to eventually be usable without limitation in a production environment. The first step was to merge the previous work on the Chrome Reactive Inspector by Waqas Abbas and Pradeep Baradur. We then target the need for more detailed information about individual nodes of the Dependency Graph and some types of applications exceeding the performance and User Interface limitations of the Chrome Reactive Inspector to cover a wider range of applications. In addition we examine the extension's User Interface in general in order to present a sound look, reduce the time a user requires to learn how to use it and, eventually, to increase their speed performing debugging tasks using the extension.


\section{Our Contribution}

In the last two sections, we presented a brief overview of Function Reactive Programming, introduced the Chrome Reactive Inspector, its origin and the scope of this thesis.
In summary, we make the following contributions:

\begin{itemize}
	\item Merging previous work on the Chrome Reactive Inspector.
	\item Reworking the User Interface, including the reduction of cognitive load on the user, to better support them in performing debugging tasks.
	\item Providing additional details to the Dependency Graph by connecting nodes with their corresponding lines of code.
	\item Reducing computational resource consumption when working with applications that contain rapidly updated Observables
	\item Providing a baseline for additional development targeting excessively created Observables.
\end{itemize}

\section{Outline}
In chapter \ref{ch:StateOfTheArt} we explain Reactive Programming in general and the two JavaScript libraries RxJS and BaconJS that the Chrome Reactive Inspector currently supports in detail. We also examine existing tools for debugging Reactive Systems and the reasons traditional debuggers are not suitable. In addition, we present the previous work on the Chrome Reactive Inspector and picture RxFiddle which is the main competitor of the Chrome Reactive Inspector.
Our contributions and reasoning behind choosing a specific approach over other possible solutions is explained in chapter \ref{ch:Contribution}.
Chapter \ref{ch:Evaluation} contains verification of our claims and implementations, as well as, results of the evaluation of the most important features in CRI for a set of test applications.
In chapter \ref{ch:Future} we summarize  We also point out possible targets of further development to increase the applicability of CRI.
