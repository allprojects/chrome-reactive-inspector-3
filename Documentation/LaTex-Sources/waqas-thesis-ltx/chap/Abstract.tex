Over the last few years, reactive programming has proved to be useful in several domains such as GUIs, animations, web applications, robotics and sensor networks. Reactive programming has been preferred over traditional techniques to implement reactive systems because it makes applications more comprehensible, composable, easy to understand and less error-prone. Unfortunately, it is hard to debug reactive code because there is no proper tool-support available. Traditional debugging tools are engineered for procedural programming and are not suited for debugging reactive applications. The goal of this thesis is to improve the debugging process of reactive systems in the web domain.

In this thesis, we present a special debugger as an extension to Google Chrome DevTools for debugging reactive applications based on JavaScript reactive libraries (\textbf{RxJS} and \textbf{Bacon.js}), called \textbf{Chrome Reactive Inspector (CRI)}. RxJs and Bacon.js, Javascript libraries for asynchronous data streams, bring together the best from functional and reactive programming concepts to the web domain. The debugger is an implementation of \textbf{RP Debugging} in the web domain where the dependency graph is used to model reactive applications.
The system presented in this thesis makes it possible to debug a specific point back-in-time, as the developer can navigate history back and forth freely. A domain-specific query language provides direct access to the graph history so that one can jump to points of interest. This query language also enables programmers to set reactive-programming-specific breakpoints. These are breakpoints which interrupt execution when a query provided by the developer matches a given criteria.

We present a preliminary evaluation based on various sample applications taken from the internet. The evaluation shows that the debugger helps in the understanding of reactive systems. It also enables bugs to be found quickly by outperforming traditional debuggers as it directly supports abstractions of reactive libraries.