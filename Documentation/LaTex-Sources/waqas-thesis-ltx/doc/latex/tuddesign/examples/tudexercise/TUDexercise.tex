\documentclass[twoside,colorbacktitle,accentcolor=tud0b]{tudexercise}
\usepackage{ngerman}

\newcommand{\unit}[1]{{\rm\,#1}}

\title{"Ubung zur Vorlesung\linebreak[1] Physik f"ur ET/IT und IKT\linebreak[1] Prof. Dr. H. Wipf}
\subtitle{Sommersemester 2004}
\subsubtitle{"Ubungsblatt \arabic{section}}


\begin{document}
  \begin{examheader}
    \textmb{"Ubung zur Vorlesung\linebreak[1] Physik f"ur ET/IT und IKT}
    \examheaderdefault
  \end{examheader}
\setcounter{section}{9}
\maketitle
  \subsection{(m"ogliche Pr"asenzaufgabe)}
    Die Temperatur $T_1$ eines Zimmers wird durch Heizen auf die Temperatur $T_2$ erh"oht. Um
    welchen Wert "andert sich dadurch die gesamte kinetische Translationsenergie $E_k$ der
    Luftmolek"ule innerhalb des Zimmers?
  \subsection{(m"ogliche Pr"asenzaufgabe)}
    Eine mit Strom betriebene Kompressor-W"armepumpe mit (idealem Carnotschen Wirkungsgrad
    $\eta_{WP}$ wird zum Heizen eines Hauses benutzt. Dazu soll Wasser mit der Temperatur
    $T=40\unit{^\circ C}$ in die Rohre einer Fu"sbodenheizung geleitet werden, wo dann pro
    Zeiteinheit die W"arme $\dot{q}=3\unit{kW}$ zum Heizen abgegeben wird. Ein Teil dieser W"arme
    wird von der Au"senluft genommen (Fachausdruck "`gestohlen"').
    \begin{enumerate}
      \item Welche Leistung $P_1$ muss der Kompressor aufbringen, wenn die Temperatur der
      Aus"senluft $T_1=5\unit{^\circ C}$ betr"agt?
      \item Welche Leistung $P_2$ muss de Kompressor bei einer Au"sentemperatur
      $T_1=20\unit{^\circ C}$ leisten? Lohnt sich dann noch einer energetisch betrachtet der Einsatz
      der W"armepumpe, wenn zur Stromerzeugung fossile Energietr"ager (Kohle, Erd"ol, Erdgas)
      verwendet werden und f"ur die Stromerzeugung der Vergleichsweise hohe Wirkungsgrad
      \hbox{$\eta_{Strom}=0,4$} angenommen wird? 
    \end{enumerate}
  \subsection{(m"ogliche Pr"asenzaufgabe)}
    Welcher Wert m"usste die Masse $m$ zweier positiver (oder negativer) Elementarladungen
    $e=1,602\cdot10^{19}\unit{As}$ haben, wenn die anziehende Gravitatonwechselwirkung die
    absto"sende Coulombwechselwirkung gerade kompensieren soll? Wie verh"alt sich die Masse $m$ zur
    Masse eines Elektrons ($m_e=9,11\cdot 10^{31}\unit{kg}$)?
  \subsection{}
    Wasser habe die spez. W"armekapazit"at $C=4,18\unit{J/(g\cdot K)}$ und die Dichte
    $\rho=1\unit{g/cm^3}$. Wegen der geringen thermischen Ausdehnung kann der Unterschied zwischen
    der W"armekapazit"at bei konstantem Volumen und bei konstantem Druck vernachl"assigt werden;
    ebenso sei die Temperaturabh"angigkeit der W"armekapazit"at und der Dichte vernachl"assigt.
    F"ur die Beantwortung der folgenden Frage sollen "Anderungen der Entropie $S$ mit Hilfe der
    Entropiedefinition $dS=\delta Q/T$ berechnet werden, wobei $\delta Q$ eine (infinitesimal
    kleine) zugeführte W"arme ist.
    \begin{enumerate}
      \item Um welchen Wert $S_{12}$ "andert sich die Entropie von Wasser mit dem Volumen
      $V_0=1\unit{l}$, wenn das Wasser von $T_1=20\unit{^\circ C}$ auf $T_2=80\unit{^\circ C}$
      erw"armt wird?
      \item Ineinem Dewar befindet sich Wasser mit der Temperatur $T_1=20\unit{^\circ C}$ und dem
      Volumen $V_1=4\unit{l}$ (das Dewar verhindert eine W"armeabgabe an die Umgebung, deine
      W"armekapazit"at soll hier vernachl"assigt werden). Es wird hei"ses Wasser mit der
      Temperatur $T_2=80\unit{^\circ C}$ und dem Volumen $V_2=2\unit{l}$ dazugegeben. Welchen Wert
      hat die Mischtemperatur $T_M$?
      \item Um welchen Wert $\Delta S$ erh"oht sich die Entropie der Welt durch diese Mischung?
    \end{enumerate}
\clearpage
  \subsection{}
    Die Temperatur der Sonnenoberfl"ache betr"agt $T_S=6000\unit{^\circ C}$. Welche Temperatur
    $T_E$ erwartet man damit f"ur die Erde unter folgenden Annahmen:
    \begin{itemize}
      \item Erde und Sonne sind schwarze Strahler.
      \item Die Erde habe eine Homogene Temperatur (z.B. kein Unterschied Tag-Nacht)
      \item Ratioaktive und alle sonstigen Prozesse, die eine zus"atzliche Erw"armung der Erde
      bewirken, sind zu vernachl"assigen.
      \item Der Sonnenradius betr"agt $R_S=6,96\cdot10^6\unit{m}$.
      \item Der Abstand der Erde von der Sonne betr"agt $a=1,495\cdot10^{11}\unit{m}$
    \end{itemize}
  \subsection{}
    Auf der $X$-Achse eines kartesischen Koordinatensystems befindet sich an der Stelle $x_1=-a$
    die negative Punktladung $Q_1=-Q$ und an der Stelle $x_2=+a$ die positive Ladung $Q_2=+Q$.
    \begin{enumerate}
      \item Berechnen Sie das elektrisch Feld $E_1$ l"angs der $X$-Achse als Funktion von $x$ und
      das elektrische Feld $E_2$ l"angs der $Y$-Achse als Funktion von $y$. In welcher Richtung
      zeigen die Felder?
      \item Welche Abh"angigkeit von $x$ bzw. $y$ haben $E_1$ und $E_2$ f"ur die Grenzf"allte
      $x\gg a$ und $y\gg a$? Es handelt sich hierbei um das "`Fernfeld eines elektrischen Dipols"'
      mit dem Dipolmoment $2aQ$.
    \end{enumerate}
\end{document}
