\documentclass[twoside,colorback,accentcolor=tud4c,11pt]{tudreport}
\usepackage{ngerman}

\usepackage[stable]{footmisc}
\usepackage[ngerman,pdfview=FitH,pdfstartview=FitV]{hyperref}

\usepackage{booktabs}
\usepackage{multirow}
\usepackage{longtable}

\newlength{\longtablewidth}
\setlength{\longtablewidth}{0.675\linewidth}

\title{Die fabelhaften Benner-Boys pr"asentieren
  stolz das TUD-\\Corporate-Design f"ur {\LaTeX}}
\subtitle{Clemens v. Loewenich\\Johannes Werner}
\subsubtitle{email: \textaccent{tud-design@pro-kevin.de}}
\uppertitleback{(\textaccent{\textbackslash uppertitleback})}
\lowertitleback{(\textaccent{\textbackslash lowertitleback})\hfill\today}
\institution{Speerspitze der Elite\\
    Kompetenzcenter der Leuchtt"urme\\
    Institut f"ur Angewandte Festkernphysik}
\dedication{Hier ist gen"ugend Platz\\
  f"ur eine Widmung (\textaccent{\textbackslash dedication}).\\
  \strut\\
  F"ur Annelore Schmidt\\
  aus dem Referat Kommunikation.\\
  Sie hat immer ein offenes Ohr\\
  f"ur unsere Fragen und Anregungen.}

\begin{document}
\maketitle
\begin{abstract}
   Im naturwissenschaftlichen Bereich wird viel mit {\LaTeX} gearbeitet. Wir bilden
   da keine Ausnahme. Das Referat Kommunikation arbeitet zwar an einer Dokumentvorlage f"ur
   {\LaTeX}, das hat uns jedoch nicht davon abgehalten, die Sache selbst in die Hand zu nehmen und
   eine eigene \textaccent{documentclass} zu erstellen.
   F"ur R"uckfragen stehen wir unter der Adresse \textaccentcolor{tud-design@pro-kevin.de}
   gerne zur Verf"ugung. \textbf{An diese Adresse bitte auch "Anderungen f"ur das Layout
   schicken, damit wir diese einpflegen k"onnen.}

   Diese Dokumentation besteht aus zwei gro"sen Abschnitten. Der erste Teil,
   die Einleitung (Kapitel~\ref{chap:Einleitung}) stellt das {\LaTeX}-Paket kurz vor und
   erl"autert noch einmal verschiedene Vorgaben des Corporate Designs. Im
   zweiten Teil (Kapitel~\ref{chap:tudreport}-\ref{chap:tudbeamer}) werden dann die
   einzelnen Klassen vorgestellt. Hier werden die Klassenoptionen sowie die
   Befehle und Umgebungen, die eingef"uhrt oder angepasst wurden, beschrieben.
   
   Beispiele zu den einzelnen Klassen findet man im Dokumentationsverzeichnis
   des Paketes.

\end{abstract}  

\tableofcontents

\chapter{\texorpdfstring{Ein paar Worte zur \LaTeX{}-Vorlage\footnote[1]{\glqq Die fabelhaften 
       Benner-Boys\grqq, sowie \glqq Die Benner-Boys von Stube 111\grqq{} sind ein eingetragenes
       Warenzeichen der \textaccent{AG Benner}.}}{Ein paar Worte zur
       \LaTeX{}-Vorlage}}\label{chap:Einleitung}
  
  \section{Einf"uhrung}
   Im naturwissenschaftlichen Bereich wird viel mit {\LaTeX} gearbeitet. Wir bilden 
   da keine Ausnahme. Das Referat Kommunikation arbeitet zwar an einer Dokumentvorlage f"ur
   {\LaTeX}, das hat uns jedoch nicht davon abgehalten, die Sache selbst in die Hand zu nehmen und
   eine eigene \textaccent{documentclass} zu erstellen.
   F"ur R"uckfragen stehen wir unter der Adresse \textaccentcolor{tud-design@pro-kevin.de}
   gerne zur Verf"ugung. \textbf{An diese Adresse bitte auch "Anderungen f"ur das Layout
   schicken, damit wir diese einpflegen k"onnen.}

  \section{Bekannte Fehler}
 
   \begin{itemize}\itemsep-0.5ex
    \item \textaccent{\textbackslash thanks} bzw.\ Fu"snoten auf der Titelseite funktionieren
      nicht. Deshalb werden sie, soweit m"oglich, ignoriert.
      Abgesehen davon sind sie nicht erlaubt.
    \item Es existiert noch keine ordentliche Vorlage f"ur Texte, wir 
     wissen also noch nicht, ob das so alles OK ist
   \end{itemize}
   Wer Fehler findet sollte sie uns melden (am besten mit dem Text zusammen, der
   den Fehler erzeugt, zuschicken).

  \section{Ben"otigte Pakete}
    Es werden folgende Pakete ben"otigt:
   \begin{itemize}\itemsep-0.5ex
     \accentfont
     \item scrreprt   \textnormal{(KOMA-Klasse f"ur die Dokumentenklasse} tudreport\textnormal{)}
     \item scrbook    \textnormal{(KOMA-Klasse f"ur die Dokumentenklasse} tudreport \textnormal{mit der Option} book\textnormal{)}
     \item scrartcl   \textnormal{(KOMA-Klasse f"ur die Dokumentenklasse} tudreport \textnormal{mit der Option} article\textnormal{)}
     \item scrlttr2   \textnormal{(KOMA-Klasse f"ur die Dokumentenklasse} tudletter\textnormal{)}
     \item beamer     \textnormal{(Klasse f"ur die Dokumentenklasse} tudbeamer\textnormal{)}
     \item fontenc    \textnormal{mit} OT1 \textnormal{und} T1 \textnormal{Codierung}
     \item fix-cm     computer modern \textnormal{Schriftart in} T1 \textnormal{Codierung}
     \item mathdesign \textnormal{(f"ur Formelsatz)}
     \item utopia     \textnormal{(Schriftart f"ur Formelsatz)}
     \item textcomp   \textnormal{(f"ur Sonderzeichen)}
     \item geometry
     \item titlesec
     \item titletoc   \textnormal{(f"ur Option} linedtoc\textnormal{)}
     \item fancyhdr
     \item graphicx
     \item xcolor
     \item xkeyval
     \item eso-pic
     \item afterpage   \textnormal{(f"ur die Dokumentenklasse} tudletter \textnormal{und} tudexercise \textnormal{)}
   \end{itemize}
  
  \section{"Ubersicht "uber alle TUD-Klassen}

  \begin{itemize}\itemsep-0.5ex
    \item \textaccent{tudreport} f"ur Berichte, Studien-, Diplom- und Doktorarbeiten,
    Kapitel~\ref{chap:tudreport}
    \item \textaccent{tudexercise} f"ur "Ubungen, Kapitel~\ref{chap:tudexercise}
    \item \textaccent{tudposter} f"ur Poster und Aush"ange, Kapitel~\ref{chap:tudposter}
    \item \textaccent{tudletter} f"ur Briefe, Kapitel~\ref{chap:tudletter}
    \item \textaccent{tudbeamer} f"ur Pr"asentationen, Kapitel~\ref{chap:tudbeamer}
  \end{itemize}

  Des weiteren existiert auch eine Klasse \textaccent{tudthesis}, die zur
  Anfertigung von Doktor-, Diplom-, MSc- und  BSc-Arbeiten vorgeschlagen
  wird. Sie ist eine Erweiterung der nicht mehr weiterentwickelten Klasse
  \textaccent{tudphysik}. Die Dokumentation ist im Paket selbst enthalten.

  \section{Die Schriftarten an der TU Darmstadt}

    Folgendes ist aus dem \textaccent{Corporate Design Handbuch} (1. Fassung August 2007)
    entnommen. Es sollen nur noch die hier angesprochenen Schriftarten verwendet
    werden. Die einzige Ausnahme bilden Pr"asentationen, bei diesen schl"agt
    das Referat f"ur Kommunikation die Schriftart \textaccent{Arial} vor,
    vor allem, weil sie auf vielen Systemen standardm"a"sig vorhanden ist.
 
  \subsection{Frontpage}
    Die Frontpage (bzw.\,Frutiger) wird ausschlie"slich f"ur Headlines, Subheadlines und Bildunterschriften
    verwendet. In Flie"stexten wird sie nicht verwendet.

    \paragraph{Frontpage Regular}
      \hfill%
      \begin{minipage}{\textwidth-\the\parindent}
        \sublinefont\normalsize
        \noindent
        abcdefghijklmnopqrstuvwxyz"a"o"u\\
        ABCDEFGHIJKLMNOPQRSTUVWXYZ"A"O"U\\
        1234567890,.;:!?\glqq\grqq{}@\texteuro\%/(\&)--
      \end{minipage}
      \begin{itemize}
        \item Subline
        \item Bildunterschrift
        \item Beschriftung von Grafiken
        \item Fachbereichs-- und Institutsbezeichnungen
      \end{itemize}

    \paragraph{Frontpage Medium}
      \hfill%
      \begin{minipage}{\textwidth-\the\parindent}
        \headfont\normalsize
        \noindent
        abcdefghijklmnopqrstuvwxyz"a"o"u\\
        ABCDEFGHIJKLMNOPQRSTUVWXYZ"A"O"U\\
        1234567890,.;:!?\glqq\grqq{}@\texteuro\%/(\&)--
      \end{minipage}
      \begin{itemize}
        \item Headline
        \item Subheadline
        \item Paginierung
      \end{itemize}

  \subsection{Charter}
    Die Charter wird als Flie"stext-- und Kommunikationsschrift eingesetzt.
    Briefe und Faxe sowie Flie"stext in Publikationen werden damit geschrieben.
    Um Inhalte hervorzuheben, kann der Bold-- oder Kursiv--Schnitt der Charter eingesetzt werden.
    Der Zeilenabstand des Flie"stextes entspricht 121\%\footnote{Stellt sich bei genauerer 
    Betrachtung als bedingt richtig heraus. Es ist nicht ersichtlich, wie die 121\%
    renormiert werden m"ussen, um die Vorgaben des \textaccent{Corporate
    Design Handbuchs} zu
    erf"ullen.} der Schriftgr"o"se.
    \paragraph{Charter Regular}
      \hfill%
      \begin{minipage}{\textwidth-\the\parindent}
        \normalfont\normalsize
        \noindent
        abcdefghijklmnopqrstuvwxyz"a"o"u\\
        ABCDEFGHIJKLMNOPQRSTUVWXYZ"A"O"U\\
        1234567890,.;:!?\glqq\grqq{}@\texteuro\%/(\&)--
      \end{minipage}
      \begin{itemize}
        \item Flie"stext
      \end{itemize}

    \paragraph{Charter Italic}
      \hfill%
      \begin{minipage}{\textwidth-\the\parindent}
        \it
        \noindent
        abcdefghijklmnopqrstuvwxyz"a"o"u\\
        ABCDEFGHIJKLMNOPQRSTUVWXYZ"A"O"U\\
        1234567890,.;:!?\glqq\grqq{}@\texteuro\%/(\&)--
      \end{minipage}
      \begin{itemize}
        \item Hervorhebung im Flie"stext
        \item Zitate
        \item "Ubersetzungen
      \end{itemize}

    \paragraph{Charter Bold}
      \hfill%
      \begin{minipage}{\textwidth-\the\parindent}
        \bf
        \noindent
        abcdefghijklmnopqrstuvwxyz"a"o"u\\
        ABCDEFGHIJKLMNOPQRSTUVWXYZ"A"O"U\\
        1234567890,.;:!?\glqq\grqq{}@\texteuro\%/(\&)--
      \end{minipage}
      \begin{itemize}
        \item Hervorhebung im Flie"stext
      \end{itemize}
    
  \subsection{Stafford}
    Stafford (bzw.\,Rockwell) wird als Auszeichnungs- und Informationsschrift eingesetzt.
    Zusatzinformationen werden durch Stafford ausgezeichnet.
    \paragraph{Stafford}
      \hfill%
      \begin{minipage}{\textwidth-\the\parindent}
        \accentfont
        \noindent
        abcdefghijklmnopqrstuvwxyz"a"o"u\\
        ABCDEFGHIJKLMNOPQRSTUVWXYZ"A"O"U\\
        1234567890,.;:!?\glqq\grqq{}@ \texteuro\%/(\&)--
      \end{minipage}
      \begin{itemize}
        \item Beschriftung von Grafiken
        \item Impressum
        \item Absender
        \item Marginalien
        \item Addressinformationen
      \end{itemize}
  
  \section{Die Farben an der TU Darmstadt}
    
    An der TU Darmstadt sollen nur noch die im Corporate Design Handbuch\footnote{1.
    Auflage, 10.08.2007} definierten Farbwerte verwendet werden. Dort hei"st es:
    \begin{quote}
     Die Farbwerte sind f"ur alle Produktionsverfahren und Medien g"ultig
     und bindend. Abweichungen sind \emph{nicht} zul"assig.

     Pro Publikation darf jeweils nur eine Akzentfarbe aus dem Farbschema
     der Technischen Universit"at verwendet werden. Bei Grafiken und Tabellen,
     f"ur die mehr als eine Farbe ben"otigt wird, k"onnen alle Farben des
     Farbschemas benutzt werden.
    \end{quote}
  Diese Forderung bleibt jedoch in der Realit"at schwer zu erf"ullen, da
  hierf"ur kalibrierte Ausgabeger"ate ben"otigt werden.

  Alle zul"assigen Farben sind in der Datei \textaccent{tudcolors.def}
  definiert. Sie lassen sich "uber das Pr"afix \textaccent{tud} und den Farbcode
  aus dem Corporate Design Handbuch ansprechen, z.\ B.\ \textcolor{tud2a}{\textbackslash
  textcolor\{tud2b\}\{<text>\}}. Die Ausnahmen sind \textaccent{black} und \textaccent{white}.
  Ist die Akzentfarbe zu hell f"ur Text, wird stattdessen die Farbe Schwarz verwendet und
  eine Warnung ausgegeben.
  
  \paragraph{Hinweis zum Paket \texorpdfstring{\textaccent{hyperref}}{hyperref}}
    Um dem Farbschema des Corporate Designs gerecht zu werden, sind Farblinks und Rahmen um Links
    ausgeschaltet. Dar"uber hinaus sind die Linkfarben auf \textaccent{tudtextaccent}
    (siehe~\ref{sec_befehle}) und die Rahmenfarben der Links auf schwarz gesetzt, so dass sich beides ohne
    farbliche Probleme einschalten l"asst. Hier sei noch auf den Befehl \textaccent{\textbackslash
    texorpdfstring\{<tex string>\}\{<pdf string>\}} hingewiesen, der die meisten Fehler mit
    \textaccent{hyperref} behebt.

\chapter{Die Dokumentenklasse \texorpdfstring{\textaccent{tudreport}}{tudreport}}\label{chap:tudreport}

  \section{Optionen der Dokumentenklasse \texorpdfstring{\textaccent{tudreport}}{tudreport}}
    Es stehen neben den Optionen f"ur die \textaccent{report} Klasse noch folgende Optionen zur
    Verf"ugung:\par
    \nopagebreak
    \begin{longtable}[h]{lp{\longtablewidth}}
      \textaccent{article}         & L"adt die \textaccent{scrartcl} Klasse
          und ersetzt die Option \textaccent{nopartpage} und \textaccent{noheadingspace}.\\
      \textaccent{book}      & l"adt die \textaccent{scrbook}-Klasse anstatt der
          \textaccent{scrreprt}-Klasse\\
      \textaccent{accentcolor=$<$color$>$} & Setzt die Akzentfarbe
          \textaccent{tudaccent} auf die Farbe \textaccent{$<$color$>$} und die Farbe
          \textaccent{tudtextaccent} entsprechend (siehe Kap.~\ref{sec_befehle}). Der  Farbname von
          \textaccent{$<$color$>$} setzt sich aus dem Prefix \textaccent{tud} und dem Farbcode
          zusammen. Ausnahmen bilden \textaccent{black} und \textaccent{white}. (default:
          \textcolor{tud0b}{\accentfont tud0b})\\ 
          &Beispiel: \textaccent{\textbackslash documentclass$[$accentcolor=\textcolor{tud9a}{tud9a}$]$
            $\{$tudreport$\}$ }\\
      \textaccent{colortitle}      & setzt den Titel in der Farbe \textaccent{tudtextaccent}
          (siehe Kap.~\ref{sec_befehle}). Diese Option sollte nur in
          Sonderf"allen verwendet werden. (wird bei Option \textaccent{colorbacktitle} ignoriert)\\
      \textaccent{colorbacktitle}  & hinterlegt den Titel mit der Akzentfarbe\\
      \textaccent{colorback}       & hinterlegt die Titelseite unter dem Titel mit der Akzentfarbe\\
          &(wird bei Option \textaccent{colorbacktitle} ignoriert)\\
      \textaccent{inverttitle}     & setzt, wenn es erlaubt ist, die Textfarbe des Titels auf Wei"s.\\
      \textaccent{inverttitlerule} & setzt, wenn es erlaubt ist, die Farbe der untersten
          Begrenzungslinie des Titels auf Wei"s\\
      \textaccent{blackrule}       & rein schwarze Identit"atsleiste\\
      \textaccent{longdoc}         & Diese Option ist f"ur lange Dokumente
        gedacht. Es wird der Seitenstil von \textaccent{plain} auf
        \textaccent{headings} gesetzt (n"ahere Erleuterung siehe unten). Impliziert die Optionen
        \textaccent{twoside} und \textaccent{openright}.\\
%        \textaccent{twoside}, \textaccent{openright} und \textaccent{bigchapter}.\\
      \textaccent{bigchapter}     & Setzt \textaccent{\textbackslash chapter} ohne die
        Begrenzungslinien und in doppelter Schriftgr"o"se der restlichen "Uberschriften.\\
      \textaccent{nopartpage}      & Beginnt den Text direkt unter der Begrenzungslinie und nicht
        auf einer neuen Seite.\\
      \textaccent{nochapterpage}   & entf"allt, wird mit \textaccent{article}
	      erzwungen.\\
      \textaccent{noheadingspace}  & Reduziert den Platz um "Uberschriften auf ein Minimum.\\
      \textaccent{noresetcounter}  & Verhindert, dass die Z"ahler f"ur Gleichungen, Fu"snoten,
          Abbildungen und Tabellen zu Beginn eines neuen Kapitels zur"uckgesetzt werden. Diese
          Option imitiert damit das Verhalten der \textaccent{article} Klasse. (Ersetzt 
          \textaccent{noresetequation} und \textaccent{noresetfootnote})\\
      \textaccent{linedtoc} & F"ugt im Inhaltsvereichnis Linien wie bei den "Uberschriften f"ur
          die oberste Gliederungsebene ein.\\
      \textaccent{numbersubsubsec} & F"ugt eine Nummerierung der Ebene 
          \textaccent{\textbackslash subsubsection} ein und listet sie im Inhaltsverzeichnis.\\
      \textaccent{firstlineindent} & aktiviert die Einr"uckung der ersten
          Zeile nach neuem \textaccent{\textbackslash chapter}, neuer
	  \textaccent{\textbackslash section}, etc.\\
      \textaccent{8pt 9.5pt 10pt 11pt 12pt} & Die Schriftgr"o"se wird auf den Standard von
          \textaccent{9.5pt} gesetzt. Dar"uber hinaus gibt es die Optionen \textaccent{8pt},
          \textaccent{10pt}, \textaccent{11pt} und \textaccent{12pt}. Die Option \textaccent{9.5pt}
          muss nicht explizit gesetzt werden.\\ 
      \textaccent{pagingbar}       & Setzt die Paginierung neben die Identit"atsleiste, wie es in
          er ersten Auflage des \textaccent{Corporate Design Handbuch} vorgesehen war. (nur in
          Verbindung mit dem Seitenstil \textaccent{plain})\\
      \textaccent{marginparwidth=<num>} & Setzt die Breite und den Separator f"ur
          Marginalien. Die Marginalienbreite wird auf <num> Rasterspaltenbreiten gesetzt.\\
      \textaccent{paper=<papersize>} & Setzt die Papiergr"o"se. Erlaubt sind f"ur
          \textaccent{<papersize>} \textaccent{a4report} und \textaccent{a5report} (mit Binderand).
          Die Papiergr"o"sen \textaccent{a4} und \textaccent{a5} werden als \textaccent{a4report}
          bzw.\ \textaccent{a5report} interpretiert. Standard ist \textaccent{a4report}.
    \end{longtable}

    Alle anderen Klassen-Optionen werden direkt an \textaccent{scrreprt} weiter gereicht.% Einzige
%    Ausnahme bildet die Option \textaccent{paper}, da \textaccent{paper=a4paper}
%    fest vorgegeben ist.

    \paragraph{Hinweis zum Farbschema}
    Ist die Akzentfarbe zu hell f"ur Text, wird statt ihrer die Farbe Schwarz verwendet und
    eine Warnung ausgegeben. Die {\LaTeX} Klasse \textaccent{tudreport} setzt die Textfarbe des
    Titels selbsts"andig auf Wei"s, wenn Schwarz nicht erlaubt ist und der Titel mit der
    Akzentfarbe hinterlegt ist. Ist in diesem Fall nur Schwarz erlaubt, wird die Option 
    \textaccent{inverttitle} ignoriert. Die Option \textaccent{inverttitlerule} wird immer
    ignoriert, au"ser die Textfarbe des Titels ist Wei"s und ein Titelbild ist vorhanden.

    \paragraph{Hinweis zu den Seitenstilen}
    Die Seitenstile, die {\LaTeX} von Hause aus kennt, werden in der \textaccent{tudreport} Klasse
    neu definiert. Der Seitenstil \textaccent{empty} beinhaltet nur die Identit"atsleiste und die
    untere Begrenzungslinie. Beim Seitenstil \textaccent{plain} kommt die Seitenzahl
    standardm"a"sig unter die Textbegrenzung. Dabei wird zwischen den Klassenoptionen
    \textaccent{oneside} und \textaccent{twoside} unterschieden. Im ersten Fall wird die Seitenzahl
    immer rechts gesetzt, im andern am "au"seren Rand, d.h abwechselnd rechts und links. Beim
    Seitenstil \textaccent{headings} werden links bzw.\ am inneren Rand unterhalb der Begrenzungslinie
    zus"atzlich Kapitelinformationen gesetzt. Dabei wird auf die Vorgaben der
    \textaccent{scrreprt} Klasse zur"uckgegriffen. Neu dazu gekommen ist der Seitenstil
    \textaccent{realempty}, der eine komplett leere Seite liefert. Alle Seitenstile sind nur f"ur
    den \textbf{internen Gebrauch} gedacht und sollten nicht direkt angesprochen werden.
    
    Wenn Interesse besteht, kann die Fu"szeile angepasst werden. Dazu steht der Seitenstil
    \textaccent{myheadings} zur Verf"ugung. Mit den Befehlen \textaccent{\textbackslash
    mymarkright}, \textaccent{\textbackslash mymarkleft}, \textaccent{\textbackslash mymarkboth}
    und \textaccent{\textbackslash mymarkcenter} kann der Inhalt der Fu"szeile analog zu den
    Befehlen \textaccent{\textbackslash markright}, \textaccent{\textbackslash markleft},
    \textaccent{\textbackslash markboth} und \textaccent{\textbackslash markcenter} ge"andert
    werden.

    Die Standardeinstellung ist \textaccent{plain}. Im Fall der Klassenoption \textaccent{longdoc}
    wird \textaccent{headings} verwendet.
    \begin{itemize}\itemsep-1ex
      \item \textaccent{realempty}
      \item \textaccent{empty}
      \item \textaccent{plain}
      \item \textaccent{headings}
      \item\textaccent{myheadings}
    \end{itemize}
    
    \paragraph{Typographische Anmerkungen}
    Die Schriftgr"o"se \textaccent{9.5pt} ist vielen Leuten etwas zu klein,
    die Zeilen des Flie"stextes werden dadurch n"amlich recht lang.
    Als Faustregel kann man sagen, dass Zeilen eigentlich k"urzer als drei
    vollst"andige Alphabete sein sollten, um gut
    lesbar zu sein. Bei Verwendung kleiner Schriftgr"o"sen kann es sinnvoll
    sein, den Text zwei- oder mehrspaltig zu setzen\footnote{Abbildungen sind
    dann auch nur eine Spalte breit. Sollen sie sich "uber die gesamte Breite
    erstrecken, so muss die Umgebung \{figure*\} verwendet
    werden}, um die Lesbarkeit zu
    verbessern. Man sollte allerdings beachten, dass sich Zeilen mit weniger 
    als etwa 40 Zeichen nur schwer zu einem ordentlichen Blocksatz verarbeiten
    lassen. Man kann jedoch durch Erzwingen von Silbentrennungen in
    Worten \LaTeX{} bei der Erstellung eines gut gesetzten Dokumentes helfen.
    Hilft dies alles nichts, so sollte man sich "uberlegen, ob man den
    Text etwas umformuliert, um auf diese Weise ein ordentliches Dokument zu
    erhalten. Selbstverst"andlich sind diese Feinarbeiten erst ganz am Ende
    des Projektes sinnvoll, da ansonsten "Anderungen weiter vorne im Text die
    geleistete Arbeit zunichte machen k"onnen. Dieses Dokument verwendet "ubrigens die
    Schriftgr"o"se~\textaccent{\the\fontdimen6\font}.

    Zeilenabst"ande sind in vielen Fachbereichen ein leidiges Thema. H"aufig
    wird ein \glqq eineinhalbfacher\grqq{} Zeilenabstand vorgeschrieben, was das
    auch immer hei"sen mag. Diese Abst"ande sind, rein typographisch
    gesehen, zu gro"s. Allerdings kann man sich gegen Vorgaben des
    Fachbereichs auch nur schwer sperren, selbst wenn das Resultat dann nicht
    gut aussieht.

  \section{Zus"atzliche Befehle}\label{sec_befehle}

    In den TUD-Klassen sind einige neue Befehle definiert. Viele von ihnen
    k"onnen in allen Klassen verwendet werden. Dadurch lassen sich relativ
    leicht ganze Abschnitte aus einem Dokument in ein anderes "ubertragen. Im
    Folgenden eine "Ubersicht "uber alle neuen Befehle.

    \begin{longtable}[h]{lp{\longtablewidth}}
      \textaccent{tudaccent}& "Uber diesen Farbnamen kann auf die Akzentfarbe zugegriffen werden.\\
          & Beispiel:
          {\fboxrule1.5pt\fboxsep1pt\fcolorbox{tudaccent}{white}{\textaccent{\textbackslash 
          fcolorbox$\{$tudaccent$\}\{$white$\}\{<$text$>\}$}}}\\
      \textaccent{tudtextaccent}& "Uber diesen Farbnamen kann auf die Akzentfarbe f"ur Text
          zugegriffen werden. Sie entspricht \textaccent{tudaccent} wenn dies zul"assig ist,
          ansonsten \textaccent{black}. In diesem Fall gibt {\LaTeX} eine Warnmeldung aus.
          \textbf{Hinweis}: Diese Farbe sollte nur dann verwendet werden, wenn
          \textaccent{\textbackslash textaccentcolor\{$<$text$>$\}} oder
          \textaccent{\textbackslash begin\{accentcolor\}}
          \textaccent{$<$text$>$}
          \textaccent{\textbackslash end\{accentcolor\}}
          nicht m"oglich ist.\\
      \textaccent{\textbackslash acdefault} & liefert die Schriftbezeichnung der
          Auszeichnungsschrift (Stafford)\\
      \textaccent{\textbackslash accentfont} & Wechselt Schrifttyp zur Auszeichnungsschrift
          (sollte nicht verwendet werden)\\
      \textaccent{\textbackslash textaccent\{$<$text$>$\}} & Setzt \textaccent{$<$text$>$} in
          der Auszeichnungsschrift.
          Die entsprechende Umgebung kann mit\\
          &$\quad$\textaccent{\textbackslash begin\{accenttext\}}
          \textaccent{$<$text$>$}
          \textaccent{\textbackslash end\{accenttext\}}\\
          & gesetzt werden.\\
      \textaccent{\textbackslash textaccentcolor\{$<$text$>$\}} & Setzt
          \textaccent{$<$text$>$} in der Akzentfarbe \textaccent{tudtextaccent} und in der
          Auszeichnungsschrift.
          Dabei wird ein Fehler ausgegeben, wenn die Akzentfarbe zu hell ist.
          Die entsprechende Umgebung kann mit\\
          &$\quad$\textaccent{\textbackslash begin\{accentcolor\}}
          \textaccent{$<$text$>$}
          \textaccent{\textbackslash end\{accentcolor\}}\\
          & gesetzt werden.\\
      \textaccent{\textbackslash tudrule[$<$length$>$]} & Zeichnet eine waagrechte Trennlinie mit
          der L"ange \mbox{\textaccent{\textbackslash linewidth}} bzw.\ \textaccent{$<$length$>$}.
          Die Liniendicke h"angt von der Papiergr"o"se ab.\\
      \textaccent{\textbackslash tudgoldenrule} & Wie \textaccent{\textbackslash tudrule}
          mit einer L"ange, so dass  \textaccent{\textbackslash linewidth}
          im Goldenen Schnitt geteilt wird. (z.B. Trennlinie f"ur Fu"snoten)\\
      \textaccent{\textbackslash title\{$<$title$>$\}} & Setzt auf der Titelseite
          den Titel auf \textaccent{$<$title$>$}
          Der Titel kann mehrzeilig sein. Ist der Titel mehr als drei Zeilen lang, wird automatisch 
          eine kleinere Schriftgr"o"se verwendet.\\
      \textaccent{\textbackslash subtitle\{$<$subheader$>$\}} & Setzt in den Abschnitt unter
          dem Titel den \glqq Subhead\grqq{} auf \textaccent{$<$subheader$>$}
          Der \glqq Subhead\grqq{} kann mehrzeilig sein.\\
      \textaccent{\textbackslash subsubtitle\{$<$subline$>$\}} & Setzt in den Abschnitt unter
          dem Titel die \glqq Subline\grqq{} auf \textaccent{$<$subline$>$}
          Die \glqq Subline\grqq{} kann mehrzeilig sein.\\
      \textaccent{\textbackslash institution\{$<$institution$>$\}} & Setzt unter die Wort-Bildmarke
          (TU-Logo) die Institutsbezeichnung auf \textaccent{$<$institution$>$}. Die
          Institutsbezeichnung kann mehrzeilig sein.\\
          & Wird \textaccent{\textbackslash setinstitutionlogo} verwendet, wird
          \textaccent{\textbackslash institution} ignoriert.\\
      \textaccent{\textbackslash setinstitutionlogo[width | height]\{$<$file$>$\}} & Setzt das Logo
          \textaccent{$<$file$>$} unter die Wort-Bildmarke (TU-Logo)
          Das Argument \textaccent{$<$file$>$} wird wie von \textaccent{\textbackslash
          includegraphics$\{<$file$>\}$} behandelt. Das optionale Argument \textaccent{[width]}
          bzw.\ \textaccent{[height]} gibt an, dass das Logo $2/3$ der Breite bzw.\ H"ohe der
          Wort-Bildmarke haben soll. Die jeweils andere Gr"o"se wird angepasst. Ohne das
          optionale Argument wird die Breite als Referenzgr"o"se verwendet.\\
      \textaccent{\textbackslash settitlepicture[$<$options$>$]\{$<$file$>$\}} & Setzt das Bild 
          \textaccent{$<$file$>$} als Hintergrundbild auf die Titelseite. Dabei werden
          H"ohe und Breite an den vorhandenen Platz angepasst.
          Die Argumente \textaccent{$<$file$>$} und \textaccent{$<$options$>$} werden wie von
          \textaccent{\textbackslash includegraphics*[$<$options$>$]\{$<$file$>$\}} behandelt.\\
      \textaccent{\textbackslash printpicturesize} & Gibt den vorhandenen Platz f"ur ein Bild 
          in der linken unteren Ecke der Titelseite aus. Um Streckungen und Stauchungen des
          Titelbildes zu vermeiden, kann somit das optimale Seitenverh"altnis bestimmt werden.\\
      \textaccent{\textbackslash sponsor\{$<$sponsor$>$\}} & F"ugt eine Sponsorleise am unteren Rand
          der Titelseite mit dem Inhalt \textaccent{$<$sponsor$>$} ein.\\
      \textaccent{\textbackslash maketitle} & Generiert die Titelseite.\\
      \textaccent{abstract} & Diese Umgebung setzt eine Seite ohne Seitenzahl, auf der eine
          Zusammenfassung steht. Der Seitenz"ahler wird auf dieser Seite nicht erh"oht. Diese
          Umgebung sollte direkt nach dem Befehl \textaccent{\textbackslash maketitle} verwendet
          werden. Die Umgebung kann mit\\
          &$\quad$\textaccent{\textbackslash begin\{abstract\}}
          \textaccent{$<$text$>$}
          \textaccent{\textbackslash end\{abstract\}}\\
          &gesetzt werden.\\
      \textaccent{seclinedepth} & Dieser Z"ahler gibt die Gliederungstiefe an, bis zu der die "Uberschriften mit
          Linien versehen werden. Er wird analog zu \mbox{\textaccent{secnumdepth}} verwendet. Standard ist
          \mbox{\textaccent{\textbackslash subsubsection}}\\
%      \textaccent{\textbackslash mbseries} & setzt das Argument in medium-bold
%          (nur bei der Frontpage verwendbar).\\
%      \textaccent{\textbackslash mb} & wie \textaccent{\textbackslash
%          normalfont\textbackslash mbseries}, sollte aber nicht verwendet werden,
%          da er veraltet ist.\\
%      \textaccent{\textbackslash mathmb} & setzt das Argument im Mathe-Modus in medium-bold
%          (nur bei der Frontpage verwendbar).\\
      \textaccent{\textbackslash markcenter} & zum "Andern der mittleren Fu"szeile beim
          Seitenstil \mbox{\textaccent{headings}}\\
      \textaccent{\textbackslash mymarkright} & zum "Andern der rechten Fu"szeile beim Seitenstil
          \mbox{\textaccent{myheadings}}\\
      \textaccent{\textbackslash mymarkleft} & zum "Andern der linken Fu"szeile beim Seitenstil
          \mbox{\textaccent{myheadings}}\\
      \textaccent{\textbackslash mymarkboth} & zum "Andern der rechten und linken Fu"szeile beim
          Seitenstil \mbox{\textaccent{myheadings}}\\
      \textaccent{\textbackslash mymarkcenter} & zum "Andern der mittleren Fu"szeile beim
          Seitenstil \mbox{\textaccent{myheadings}}\\
      \textaccent{\textbackslash textwhitespace} & gibt ein \textwhitespace{} aus\\
      \textaccent{abstract} & Die Umgebung akzeptiert ein optionales Argument \textaccent{<num>}, um zwei 
	  Zusammenfassungen verschiedener Sprachen zu erm"oglichen. F"ur die erste Zusammenfassung ist \textaccent{<num>} 1,
	  f"ur die zweite 2.\\
    \end{longtable}

    Die Befehle wie \textaccent{\textbackslash author} oder \textaccent{\textbackslash date}
    existieren zwar, werden aber nicht zur Erstellung der Titelseite
    verwendet. An ihrer Stelle
    m"ussen \textaccent{\textbackslash subtitle} und \textaccent{\textbackslash subsubtitle}
    verwendet werden. Nur die Befehle \textaccent{\textbackslash title},
    \textaccent{\textbackslash uppertitleback}, \textaccent{\textbackslash lowertitleback} 
    und \textaccent{\textbackslash dedication} der KOMA-Klasse \textaccent{scrreprt} werden
    unterst"utzt.

\chapter{Die Dokumentenklasse \texorpdfstring{\textaccent{tudexercise}}{tudexercise}}\label{chap:tudexercise}
  
  Die Dokumentenklasse \textaccent{tudexercise} kann zur Erstellung von
  "Ubungsbl"attern und Tischvorlagen genutzt werden. Auf die erste Seite wird oben eine
  Kopfleiste gesetzt, in der die Identit"atsleiste, das Logo der TU Darmstadt, der
  Titel des Dokumentes und die Untertitel (\textbackslash subtitle und
  \textbackslash subsubtitle, f"ur den Namen des Dozenten bzw.\ der Dozentin,
  und das Datum) stehen. Die folgenden Seiten haben nur noch die
  Identit"atsleiste und die Fu"sleiste.
  
  Je nach dem, welche Sprache eingestellt ist (z.B. mit dem Paket \textaccent{babel} oder
  \textaccent{ngerman}), wird jedem Kapitel \textbf{Problem} oder \textbf{Aufgabe} vorangestellt.
  Dies l"a"st sich mit der Option \textaccent{nochapname} unterdr"ucken.

  \section{Optionen der Dokumentenklasse \texorpdfstring{\textaccent{tudexercise}}{tudexercise}}

  Die Klasse \textaccent{tudexercise} unterst"utzt alle Optionen der
  \textaccent{tudreport} Klasse bis auf:
  \begin{itemize}\itemsep-0.5ex
     \item \textaccent{longdoc}
     \item \textaccent{bigchapter}
     \item \textaccent{colorback}
     \item \textaccent{nopartpage} (hat keine Wirkung)
     \item \textaccent{noheadingspace} (hat keine Wirkung)
     \item \textaccent{noresetcounter} (hat keine Wirkung)
  \end{itemize}
  Es wird stattdessen die Option \textaccent{article} der Klasse
  \textaccent{tudreport} \emph{fest} gesetzt.\\[0.5\baselineskip]
  Zus"atzlich gibt es die Optionen \textaccent{solution} und \textaccent{nochapname}. Die Option \textaccent{solution}
  ersetzt den Kapitelnamen \glqq Aufgabe\grqq{} durch \glqq{} L"osung\grqq. Mit der Option \textaccent{nochapname} kann
  der Kapitelname unterdr"uckt werden.
        
  \section{Zus"atzliche Befehle}
    Wie bei \textaccent{tudreport}. Dar"uber hinaus speziell f"ur Klausuren
    gedacht gibt es:
    \begin{itemize}\itemsep-0.5ex
     \item Durch die Umgebung \textaccent{examheader} kann ein Seitenkopf angegeben werden, der ab
       der zweiten Seite ausgegeben wird. Mit einem optionalen Argument kann dieser 
       Seitenkopf ab einer beliebigen Seite gesetzt werden. Soll dies z.B. die dritte Seite sein,
       sie das so aus:
       \begin{verbatim}
  \begin{examheader}[3]
    \textmb{Vordiplom Experimentalphysik WS 2000/2001}
    \examheaderdefault
  \end{examheader}
       \end{verbatim}
     \item \textaccent{\textbackslash examheaderdefault} gibt eine Zeile mit Feldern f"ur Name,
     Vorname und siebenstellige Matrikelnummer aus (s. \textaccent{\textbackslash textwhitespace})
  \end{itemize}
    

\chapter{Die Dokumentenklasse \texorpdfstring{\textaccent{tudposter}}{tudposter}}\label{chap:tudposter}
  
  Mit der Dokumentenklasse \textaccent{tudposter} lassen sich Aush"ange und
  Konferenzposter erzeugen.

  \section{Optionen der Dokumentenklasse \texorpdfstring{\textaccent{tudposter}}{tudposter}}
        
  Die Klasse \textaccent{tudposter} unterst"utzt alle Optionen der
  \textaccent{tudreport} Klasse bis auf:
  \begin{itemize}\itemsep-0.5ex
     \item \textaccent{longdoc}
     \item \textaccent{bigchapter}
     \item \textaccent{colorback}
     \item \textaccent{8pt 9.5pt 11pt 12pt}. Die Schriftgr"o"se wird stattdessen
       mit der Papiergr"o"se skaliert.
  \end{itemize}
  Zus"atzlich sind folgende Optionen erlaubt bzw.\ verhalten sich anders:
  \begin{itemize}\itemsep-0.5ex
    \item \textaccent{paper=[a1,a2,a3,a4,a0,a0b]}, wobei die Gr"o"se
      \textaccent{a0b} Standard ist und Schnittmarken erzeugt.
    \item \textaccent{colorbacktitle} hinterlegt nur den Titel mit der Akzentfarbe.
    \item \textaccent{colorbacksubtitle} hinterlegt den Untertitel mit der Akzentfarbe,
      wenn \textaccent{colorbacktitle} ausgew"ahlt wurde.
  \end{itemize}
        
  \section{Zus"atzliche Befehle}
    Wie bei \textaccent{tudreport}.
    

    \paragraph{Typographische Anmerkungen}
    Die von uns voreingestellten Schriftgr"o"sen sind mit Sicherheit nicht f"ur
    alle Anwendungen geeignet, das ist auch nur schwer m"oglich. In diesem
    Fall m"ussen dann die Schriftgr"o"sen \emph{in der Pr"aambel des aktuellen
    Dokuments} neu definiert werden. Anleitungen
    hierzu findet man in \LaTeX-B"uchern, dem WWW und zumindest als Beispiel
    in der Datei \textaccent{tudpostr\_fonts.sty}, die man aber \emph{nicht}
    direkt editieren sollte!

    Beim Druck von Konferenzpostern des Formates A0 im HRZ der TU Darmstadt sollte die
    Papiergr"o"se a0b verwendet werden (voreingestellt), da das der Breite des
    dort verwendeten Endlospapieres entspricht. Zus"atzlich werden
    Schnittmarken erzeugt, mit deren Hilfe sich das Poster leicht auf das
    gew"unschte Format A0 zuschneiden l"asst.

    K"astchen, Schattenw"urfe, Trennlinien und "ahnliche 
    \glqq Gestaltungselemente\grqq{} sind bei einem ordentlichen Layout unn"otig.
    Das sollte man im Hinterkopf behalten, wenn man solche Elemente verwenden
    will. Allerdings gibt es bestimmt Anwendungen, bei denen so etwas nach
    reiflicher "Uberlegung eingesetzt werden kann.

    Ein Nachteil von \LaTeX{} ist, dass es kein ordentliches Zeilenraster
    verwendet. Dadurch fluchten Zeilen in unterschiedlichen Spalten nicht.

    Weitere freie M"oglichkeiten ein Poster zu erzeugen sind Programme wie scribus
    (bzw.\ scribus-ng), ein Open-Source DTP-Programm\footnote{DTP: DeskTop
    Publishing} mit gro"sem Funktionsumfang, xfig, ein vektorbasiertes
    Graphikprogramm, OpenOffice.org, die freie Office-Suite (bzw.\ das f"ur
    Angeh"orige der TU Darmstadt verf"ugbare Staroffice). Im kommerziellen Bereich werden
    Programme wie Quark und die Adobe-Suite verwendet, die einen gro"sen
    Funktionsumfang bieten. Leider ist es bei allen diesen Programmen schwer
    oder unkomfortabel, ordentliche Formeln zu setzen\footnote{Scribus hat in
    der aktuellen Version ($\ge$1.3.5) einen \LaTeX-Interpreter, der das Einbinden von Formeln
    erleichtert}.

\chapter{Die Briefklasse \texorpdfstring{\textaccent{tudletter}}{tudletter}}\label{chap:tudletter}
  \section{Optionen der Klasse \texorpdfstring{\textaccent{tudletter}}{tudletter}}
    
    Die Klasse \textaccent{tudletter} unterst"utzt zus"atzlich zu den Optionen
    der \textaccent{scrlttr2}-Klasse folgende Optionen:\par
    \nopagebreak
    \begin{longtable}[h]{lp{\longtablewidth}}
      \textaccent{accentcolor=$<$color$>$} & Setzt die Akzentfarbe
          \textaccent{tudaccent} auf die Farbe \textaccent{$<$color$>$} und die Farbe
          \textaccent{tudtextaccent} entprechend (siehe Kap.~\ref{sec_befehle}). Der  Farbname von
          \textaccent{$<$color$>$} setzt sich aus dem Prefix \textaccent{tud} und dem Farbcode
          zusammen. Ausnahmen bilden \textaccent{black} und \textaccent{white}. (default:
          \textcolor{tud0b}{\accentfont tud0b})\\ 
          &Beispiel: \textaccent{\textbackslash documentclass$[$accentcolor=\textcolor{tud9a}{tud9a}$]$
            $\{$tudletter$\}$ }\\
      \textaccent{blackrule}       & rein schwarze Indentit"atsleiste\\
      \textaccent{logo[=true|=false]\normalfont{,} nologo} & Setzt oder entfernt die Wort-Bildmarke (TUD-Logo) ab der
	      zweiten Seite. \textaccent{logo} entspricht \textaccent{logo=true}, \textaccent{nologo}
          \textaccent{logo=false}.\\
          & Bei Verwendung von \textaccent{twoside} wird \textaccent{nologo} gesetzt und
          umgekehrt.\\
      \textaccent{adr=$<$adr-file$>$} & L"adt die Konfigurationsdatei \textaccent{$<$adr-file$>$}
          bzw.\ die Konfigurationsdatei \textaccent{$<$adr-file$>$.adr}.
          N"aheres dazu in Kapitel~\ref{sec:tudletter:adr}
    \end{longtable}
  \section{Die Variablen}
    Die \textaccent{tudletter}-Klasse unterst"utzt alle Variablen der \textaccent{scrlttr2}-Klasse
    bis auf:
    \begin{itemize}\itemsep-0.5ex
      \item \textaccent{place} (Ort)
      \item \textaccent{specialmail} (Versandart)
      \item \textaccent{title} (Brieftitel)
    \end{itemize}
    Neu hinzugekommen sind:
    \begin{itemize}\itemsep-0.5ex
      \item \textaccent{frominstitution} (Institutsbezeichnung)
      \item \textaccent{shortfromname} (kurzer Absendername f"ur Standard-\textaccent{backaddress})
    \end{itemize}
    Wie diese Variablen zu setzen sind, ist der Dokumentation der \textaccent{scrlttr2}-Klasse zu
    entnehmen.

  \section{Anmerkung zur Seitennummerierung}
    Bei mehrseitigen Briefen wird unter der Textbegrenzungslinie die Seitenzahl und die Seitenzahl
    der letzten Seite in der Form \mbox{\textsf{Seite}
    \textaccent{<page>}\textsf{/}\textaccent{<last page>}} gesetzt. Die Seitenzahl der letzten
    Seite ist jedoch erst nach einem "Ubersetzungsdurchlauf bekannt. Der Brief muss deshalb zweimal
    "ubersetzt werden, "ahnlich wie bei Referenzen oder Inhaltsverzeichnissen.

  \section{Die Konfigurations-Dateien}\label{sec:tudletter:adr}
    Es gibt drei m"ogliche Konfigurationsdateien, die in einer bestimmten Reihenfolge geladen
    werden. In diesen Konfigurationsdateien k"onnen allgemeine Definitionen wie Absender stehen.
    Wichtig ist, dass diese Dateien im \LaTeX{}-Suchpfad stehen. Dabei ist es egal, ob dies das
    aktuelle Verzeichnis ist, das \textaccent{texmf}-Verzeichnis des Benutzers oder das globale
    \textaccent{texmf}-Verzeichnis.
    
    Zuerst wird die Datei \textaccent{tudletter.adr} geladen. Sie ist f"ur Definitionen gedacht,
    die f"ur alle Benutzer gleich sind (z.B. Institut), und sollte im globalen \textaccent{texmf}-Verzeichnis
    liegen. Danach wird die Konfigurationsdatei aus der Klassenoption
    \mbox{\textaccent{adr=$<$adr-file$>$}} geladen. Diese Datei ist f"ur
    pers"onliche
    Standardkonfigurationen gedacht, wie z.B. Absendername oder Telefonnummer. Zuletzt wird die
    Datei geladen, die so hei"st wie die \textaccent{.tex}-Datei, jedoch mit der Dateierweiterung
    \textaccent{.adr}. Diese Datei wird nur im aktuellen Verzeichnis gesucht.
    Zu beachten ist, dass jede Konfiguration aus einer zuvor geladenen Datei von einer danach
    geladenen "uberschrieben werden kann.

\chapter{Die Pr"asentationsklasse \texorpdfstring{\textaccent{tudbeamer}}{tudbeamer}}\label{chap:tudbeamer}

  \section{Optionen der Klasse \texorpdfstring{\textaccent{tudbeamer}}{tudbeamer}}
    
    Die Klasse \textaccent{tudbeamer} unterst"utzt zus"atzlich zu den Optionen
    der \textaccent{beamertex}-Klasse folgende Optionen:\par
    \nopagebreak
    \begin{longtable}[h]{lp{\longtablewidth}}
      \textaccent{accentcolor=$<$color$>$} & Setzt die Akzentfarbe
          \textaccent{tudaccent} auf die Farbe \textaccent{$<$color$>$} und die Farbe
          \textaccent{tudtextaccent} entprechend (siehe Kap.~\ref{sec_befehle}). Der  Farbname von
          \textaccent{$<$color$>$} setzt sich aus dem Prefix \textaccent{tud} und dem Farbcode
          zusammen. Ausnahmen bilden \textaccent{black} und \textaccent{white}. (default:
          \textcolor{tud0b}{\accentfont tud0b})\\ 
          &Beispiel: \textaccent{\textbackslash documentclass$[$accentcolor=\textcolor{tud9a}{tud9a}$]$
            $\{$tudbeamer$\}$ }\\
      \textaccent{colortitle}      & setzt den Titel in der Farbe \textaccent{tudtextaccent}
          (siehe Kap.~\ref{sec_befehle}). Diese Option sollte nur in
          Sonderf"allen verwendet werden. (wird bei Option \textaccent{colorbacktitle} ignoriert)\\
      \textaccent{colorbacktitle}  & hinterlegt den Titel mit der Akzentfarbe\\
      \textaccent{colorback}       & hinterlegt die Titelseite unter dem Titel mit der Akzentfarbe\\
          &(wird bei Option \textaccent{colorbacktitle} ignoriert)\\
      \textaccent{inverttitle}     & setzt, wenn es erlaubt ist, die Textfarbe des Titels auf Wei"s.\\
      \textaccent{inverttitlerule} & setzt, wenn es erlaubt ist, die Farbe der untersten
          Begrenzungslinie des Titels auf Wei"s\\
      \textaccent{blackrule}       & rein schwarze Indentit"atsleiste
    \end{longtable}
        
  \section{Zus"atzliche Befehle}
    
    \begin{longtable}[h]{lp{\longtablewidth}}
      Umgebung \textaccent{titlepage}& Diese Umgebung erzeugt die
      Titelseite.\\
      \textaccent{tudaccent}& "Uber diesen Farbnamen kann auf die Akzentfarbe zugegriffen werden.\\
          & Beispiel:
          {\fboxrule1.5pt\fboxsep1pt\fcolorbox{tudaccent}{white}{\textaccent{\textbackslash 
          fcolorbox$\{$tudaccent$\}\{$white$\}\{<$text$>\}$}}}\\
      \textaccent{tudtextaccent}& "Uber diesen Farbnamen kann auf die Akzentfarbe f"ur Text
          zugegriffen werden. Sie entspricht \textaccent{tudaccent} wenn dies zul"assig ist,
          ansonsten \textaccent{black}. In diesem Fall gibt {\LaTeX} eine Warnmeldung aus.
          \textbf{Hinweis}: Diese Farbe sollte nur dann verwendet werden, wenn
          \textaccent{\textbackslash textaccentcolor\{$<$text$>$\}} oder
          \textaccent{\textbackslash begin\{accentcolor\}}
          \textaccent{$<$text$>$}
          \textaccent{\textbackslash end\{accentcolor\}}
          nicht m"oglich ist.\\
      \textaccent{\textbackslash textaccentcolor\{$<$text$>$\}} & Setzt
          \textaccent{$<$text$>$} in der Akzentfarbe \textaccent{tudtextaccent}.
          Dabei wird ein Fehler ausgegeben, wenn die Akzentfarbe zu hell ist.
          Die entsprechende Umgebung kann mit\\
          &$\quad$\textaccent{\textbackslash begin\{accentcolor\}}
          \textaccent{$<$text$>$}
          \textaccent{\textbackslash end\{accentcolor\}}\\
          & gesetzt werden.\\
      \textaccent{\textbackslash logo[$<$Multiplikator$>$]\{$<$Logo$>$\}} &
          Beinhaltet das (Instituts-)Logo. Der optionale Multiplikator kann
         verwendet werden, um die Logogr"o"se anzupassen, normalerweise ist
         er auf den Wert 1 gesetzt.
    \end{longtable}
    Die Umgebung \textaccent{titlepage} ben"otigt zwei Durchl"aufe, um
    aktualisiert zu werden, "ahnlich wie Inhaltsverzeichnisse und Referenzen,
    da die ben"otigten L"angen erst nach der Erzeugung der Titelseite
    verf"ugbar werden.

  \section{Bekannte Fehler}

    Die im Corporate Design Handbuch gezeigte M"oglichkeit, eine
    titelseitenf"ullende Graphik einzubinden, funktioniert nicht. Ob dies
    jemals zufriedenstellend implementiert werden kann ist leider fraglich.
    Kreative L"osungen der Anwender sind hier gefragt.

\end{document}
