\chapter{Conclusion and Future work} \label{chap:Summary}
In this chapter, we conclude the thesis by shortly summarising the main contributions. Additionally, various ideas for future work are introduced, so that development can proceed without any break. 

\section{Final Remarks}
The developed system is based on an already existing Chrome plugin developed at Technical University Darmstadt, which already provides the functionality to visualize dependency graph, navigating through history. The contribution of this thesis is to extend these functionalities to develop an advanced debugger and to support various RxJS operators.

The system developed in this thesis allows user to search for a particular node, to gain an understanding of its dependents and dependencies. Additionally, advanced methods to navigate the history has been developed. As shown in the evaluation, advanced debugging mechanisms for reactive programming have the potential to ease the development of reactive systems. Many ideas for further improvements are introduced in the last section.

\section{Future Work}
We believe that the CRI extension can very well be improved in the future. One aspect would be to try to instrument the javascript code using other methods such as ESPRIMA\cite{esprima} due to limitations of Jalangi framework to support latest version of javascript. A survey on, how CRI helps the developers in real time to understand the reactive applications and find bugs in them, would provide further scope for development. 

Another aspect would be to support other browsers such as Firefox and Safari. Also it would be a nice idea to integrate with NodeJS\cite{nodejs} ecosystem to debug the applications. A great help would be to improve the debugger in order to make it a so-called ``edit-and-debug''. This would give the developer the possibility to change values in dependency graph and see the output on the fly. The plugin could also provide great assistance with performance analysis. It would be a very helpful to visualize the number of times a node is evaluated. This would help developers to find the bottleneck in their code. 