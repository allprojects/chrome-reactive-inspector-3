Reactive Programming is a programming paradigm designed to provide developers with the right abstractions for creating systems that uses streams of data. The dependencies are propagated by the language runtime, relieving the user boilerplate code from management dependencies. Reactive programs' execution is mostly driven by data flow, debugging such programs by traditional debuggers is hard. Due to the lack of support for the abstractions provided, developers have to use the most readily available debug tool: \textbf{printf}-debugging. The aim of this thesis is to develop a debugging tool that aids comprehension and debugging of reactive systems, by visualizing the dependencies and structure of the data flow, and the data inside the flow. 

\vspace*{0.05in}

In this thesis, we present \textbf{Chrome Reactive inspector}, a platform for visualization as well as required instrumentation for Javascript libraries (\textbf{RxJS} and \textbf{Bacon.js}).RxJs and Bacon.js libraries are used mainly when the data stream is asynchronous, and these bring the best from both functional and reactive programming concepts to the web applications. 

\vspace*{0.05in}

With the help of the system developed in this thesis, it is possible to visualize the flow of the program, step by step evolution of the program and the user can navigate the different stages of the program back and forth. We also derive an inspiration from marble diagrams that are extensively used to explain various RxJS concepts. We present first evaluation based on several examples that show how the extension provides, easier and intuitive way of understanding data streams, their origins and the way they mutate through the flow of the program.



















